%
% Copyright (C) 2012-2021 Javier Bezos http://www.texnia.com
%
% This file may be distributed and/or modified under the conditions of
% the MIT License. A version can be found at the end of this file.
%

\documentclass[a4paper,twocolumn]{article}

\title{Underlining (and more) with soulpos}
\author{Javier Bezos}
\date{2021-10-15}

\usepackage{tikz}
\usetikzlibrary{decorations.pathmorphing}

% \usepackage{soul}
\usepackage{soulpos}

\renewcommand{\ulpostolerance}{12}

\advance\textwidth16mm
\advance\oddsidemargin-8mm

\makeatletter
\newenvironment{desc}
  {\vskip2ex
   \hrule
   \trivlist\item[]}
  {\endtrivlist}
\makeatother

\begin{document}

\maketitle

This package just combines \textsf{soul} with the \texttt{savepos}
mechanism provided by the pdftex engine, so that you can create
(almost) arbitrary underlining and similar ``decorations'', including
rules, leaders and even pictures (\textsf{pgf}, \textsf{pstricks},
etc.). Unlike \textsf{soul} underlines, which are built by repeating small
elements, here each chunk of text to be underlined is a single
element.

The main drawback is obvious -- since it relies on \verb|\pdfsavepos|
two passes are necessary.  Further, to prevent exhausting the hash
table, the auxiliary file containing the information about each
underline is read only when needed, which may impact performance
negatively.\footnote{Actually, two files are created, with extensions
\texttt{upa} and \texttt{upb}.}

Internally \textsf{soul} knows to some point where a break happens,
and this information may be used to set different styles depending on
the position.

This version (1.0) does almost no checking (e.g., to warn about the
need for a new run), which is left for a later release.

The package \textsf{soul} is loaded if it has not been loaded before.

\section{Usage}

Underlining macros are defined much like in \textsf{soul}.

\begin{desc}
\verb|\ulposdef{<name>}[<options>]{<commands>}|
\end{desc}

Defines an underline as \verb|<commands>|, which is placed in a box of
width zero with either \verb|\llap| or \verb|\rlap|, as explained
below.  Typically, \verb|<commands>| will contain a rule or leaders.
If the text spans more than one line, then there will be several
chunks to be underlined (one per line).

You can use the following macros in \verb|<commands>|.

\begin{desc}
\verb|\ulwidth|
\end{desc}

The width of the text block to be underlined. A basic underlining is:
\begin{verbatim}
\ulposdef{\uline}{\rule[-.8ex]{\ulwidth}{.5pt}}
\end{verbatim}

\begin{desc}
\verb|\ifulstarttype{<type>}{<true>}{<false>}|\qquad
\verb|\ifulendtype{<type>}{<true>}{<false>}|
\end{desc}

These tests can be used in \verb|<commands>|, to set different underlining
styles depending on where the current chunk begins or ends.  Here
\verb|<type>| is: 0 if the very start or end of the underlined text, 1 if a
space, 2 if a discretionary hyphen and 3 if an explicit hyphen. See 
an example below.

\begin{desc}
\verb|\ulstarttype|\qquad\verb|\ulendtype|
\end{desc}

Macros storing the values described above, so that you can use
\LaTeX{} conditionals (or \TeX{} ones). So, to check if the end is an
hyphen, test \verb|\ulendtype>1|.

\vskip2ex
\hrule
\vskip2ex

Valid keys/values in \verb|<options>| are:

\begin{description}
\item[\texttt{xoffset}]
This key provides a simple way to fix an unpleasant effect found in
many programs when colouring the text background -- the colour starts
and ends just at the edges of the first and last characters.  Just set
the offset to a value larger than 0pt, as for example .1 em.  This
value is added ($\times$2) to \verb|\ulwidth|, but no space is added
to the text (which can be done with the \verb|gap| key).  Of course,
you can still do finer adjustments in the definition of the underline,
as shown in the samples below. You can use \verb|xoffset-start| and 
\verb|xoffset-end| to set the corresponding values separately.

\item[\texttt{gap}] It is equivalent to the outer space in
\verb|\sodef|.  This value is \textit{not} added to \verb|\ulwidth|.
You can use \verb|gap-start| and \verb|gap-end| to set the
corresponding values separately.

\item[\texttt{hyphens}] Sometimes excluding the hyphen from the
underlined text could make sense.  Default is \verb|hyphens=include|
but you can set it to \verb|hyphens=exclude|.

\item[\texttt{overdraw}] By default underlines are drawn before the
text is typeset (with \verb|\rlap|), so that they are placed behind.
However, it can be drawn after (with \verb|\llap|), on top the text,
with \verb|overdraw| or, equivalently, \verb|overdraw=true| (default
is \verb|overdraw=false|).
\end{description}

\begin{desc}
\verb|\ulpostolerance|
\end{desc}

The current algorithm is based on changes of the $y$ coordinate of
savepos and therefore presumes a regular baseline.  Any increasing or
decreasing of $y$ is considered a new chunk, but you can give a certain
tolerance with, for example:
\begin{verbatim}
\renewcommand{\ulpostolerance}{12}
\end{verbatim}

\section{Future work}

\begin{itemize}
\item Warnings. 
\item In short documents, providing an option for using the aux file.
\item Redefinable macros.
\item \texttt{offset}, including edges at line breaks.
\item \texttt{.upb} is generated at the end of the run, so it might be
used by another program to generate the decorations.  As of 1.0.
however, its syntax is likely to change and therefore unsupported.
\item Better manual.
\item Improved performance.
\item Fixing some issues in multicolumn text (it does not work if there 
are just two lines, one at the left and the other at the right).
\item Some predefined ``decorations''.
\end{itemize}

\newcommand\ulpossample[1]{%
A single #1{word}.  Now #1{a few words}.  Longer:
#1{this text spans several lines, so that you can see the behaviour of
\textsf{soulpos} where there are line breaks}.\vskip2ex\hrule\vskip2ex}

\section{Examples}

%  \def\ulpostolerance{-78}

\makeatletter
\expandafter\def\expandafter\verbatim@font\expandafter{%
  \verbatim@font\small}
\makeatother

\begin{verbatim}
\ulposdef{\ulpgfA}{%
  \raisebox{-.75ex}{%
    \begin{tikzpicture}%
    \clip (0,-1pt) rectangle (\ulwidth,1pt);
    \draw[
      color=black!40,
      line width=.7pt,
      decorate,
      decoration=
        {random steps,
         segment length=1.5mm,
         amplitude=.5pt}]
    (0,0) -- +(\ulwidth+3pt,0);
    \end{tikzpicture}}}
\end{verbatim}

\ulposdef\ulpgfA{%
  \raisebox{-.75ex}{%
    \begin{tikzpicture}%
    \clip (0,-1pt) rectangle (\ulwidth,1pt);
    \draw[
      color=black!40,
      line width=.7pt,
      decorate,
      decoration=
        {random steps,segment length=1.5mm,amplitude=.5pt}]
    (0,0) -- +(\ulwidth+3pt,0);
    \end{tikzpicture}}}

\ulpossample\ulpgfA

\begin{verbatim}
\ulposdef{\ulpgfB}{%
  \raisebox{-.75ex}{%
    \begin{tikzpicture}%
    \clip (0,-1pt) rectangle (\ulwidth,1pt);
    \draw[color=black!40,
      line width=.7pt,
      decorate,
      decoration=
        {snake,
         amplitude=.3pt,
         segment length=1mm,}]
     (0,0) -- +(\ulwidth+3pt,0);
     \end{tikzpicture}}}
\end{verbatim}

\ulposdef\ulpgfB{%
 \raisebox{-.75ex}{%
  \begin{tikzpicture}%
  \clip (0,-1pt) rectangle (\ulwidth,1pt);
  \draw[color=black!40,
   line width=.7pt,
   decorate,
   decoration=
     {snake, amplitude=.3pt,segment length=1mm,}]
   (0,0) -- +(\ulwidth+3pt,0);
   \end{tikzpicture}}}

\ulpossample\ulpgfB

\begin{verbatim}
\ulposdef{\ulpgfC}[xoffset=.15em]{%
  \ifulstarttype{0}%
    {\def\arr{|}}%
    {\def\arr{<}}%
  \ifulendtype{0}%
    {\edef\arr{\arr-|}}%
    {\edef\arr{\arr->}}%
  \raisebox{-.7ex}{%
    \tikz
    \draw[\arr,color=black!40,
          line width=1pt]
    (0,0) -- +(\ulwidth-1pt,0);}}
\end{verbatim}

\ulposdef\ulpgfC[xoffset=.15em]{%
  \ifulstarttype{0}%
    {\def\arr{|}}%
    {\def\arr{<}}%
  \ifulendtype{0}%
    {\edef\arr{\arr-|}}%
    {\edef\arr{\arr->}}%
  \raisebox{-.7ex}{%
    \tikz
    \draw[\arr,color=black!40,
          line width=1pt]
    (0,0) -- +(\ulwidth-1pt,0);}}

\ulpossample\ulpgfC

\begin{verbatim}
\ulposdef{\ulbgdD}{%
  \mbox{%
    \color{black!30}%
    \rule[-.8ex]{\ulwidth}{13pt}}}
\end{verbatim}

\ulposdef\ulbgdD{%
  \mbox{%
    \color{black!30}%
    \rule[-.8ex]{\ulwidth}{13pt}}}

\ulpossample\ulbgdD

\begin{verbatim}
\ulposdef{\ulbgdE}[xoffset=.1em]{%
  \mbox{%
    \color{black!30}%
    \rule[-.8ex]{\ulwidth}{13pt}}}
\end{verbatim}

\ulposdef\ulbgdE[xoffset=.1em]{%
  \mbox{%
    \color{black!30}%
    \rule[-.8ex]{\ulwidth}{13pt}}}

\ulpossample\ulbgdE

\begin{verbatim}
\ulposdef{\uldash}{%
  \makebox[\ulwidth]{%
    \color{blue}%
    \xleaders\hbox to.27em
      {\hss\rule[-.8ex]{.18em}{.5pt}\hss}%
    \hfill}}
\end{verbatim}

\ulposdef\uldash{%
  \makebox[\ulwidth]{%
    \color{blue}%
    \xleaders\hbox to.27em{\hss\rule[-.8ex]{.18em}{.5pt}\hss}\hfill}}

\ulpossample\uldash

\begin{verbatim}
\ulposdef{\uldot}{%
 \mbox{%
   \raisebox{-.85ex}{%
     \xleaders\hbox to.2em
       {\hss\footnotesize.\hss}\hskip\ulwidth}}}
\end{verbatim}

\ulposdef\uldot{%
 \mbox{%
   \raisebox{-.85ex}{\xleaders\hbox to.2em{\hss\footnotesize.\hss}\hskip\ulwidth}}}

\ulpossample\uldot

\begin{verbatim}
\ulposdef{\ulflag}{%
  \mbox{%
    \color{red}\rule[-.85ex]{.25\ulwidth}{1.5pt}%
    \color{yellow}\rule[-.85ex]{.5\ulwidth}{1.5pt}%
    \color{red}\rule[-.85ex]{.25\ulwidth}{1.5pt}}}
\end{verbatim}

\ulposdef\ulflag{%
  \mbox{%
    \color{red}\rule[-.85ex]{.25\ulwidth}{1.5pt}%
    \color{yellow}\rule[-.85ex]{.5\ulwidth}{1.5pt}%
    \color{red}\rule[-.85ex]{.25\ulwidth}{1.5pt}}}

\ulpossample\ulflag

\end{document}

MIT License
-----------

Permission is hereby granted, free of charge, to any person obtaining a
copy of this software and associated documentation files (the
"Software"), to deal in the Software without restriction, including
without limitation the rights to use, copy, modify, merge, publish,
distribute, sublicense, and/or sell copies of the Software, and to
permit persons to whom the Software is furnished to do so, subject to
the following conditions:

The above copyright notice and this permission notice shall be included
in all copies or substantial portions of the Software.

THE SOFTWARE IS PROVIDED "AS IS", WITHOUT WARRANTY OF ANY KIND, EXPRESS
OR IMPLIED, INCLUDING BUT NOT LIMITED TO THE WARRANTIES OF
MERCHANTABILITY, FITNESS FOR A PARTICULAR PURPOSE AND NONINFRINGEMENT.
IN NO EVENT SHALL THE AUTHORS OR COPYRIGHT HOLDERS BE LIABLE FOR ANY
CLAIM, DAMAGES OR OTHER LIABILITY, WHETHER IN AN ACTION OF CONTRACT,
TORT OR OTHERWISE, ARISING FROM, OUT OF OR IN CONNECTION WITH THE
SOFTWARE OR THE USE OR OTHER DEALINGS IN THE SOFTWARE.
